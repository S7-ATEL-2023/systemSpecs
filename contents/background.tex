\section{background}
\label{sec:background}

The first step in designing a system is to explore what options 
are available. A few of the main ways of multiplexing are: time-division multiplexing, 
frequency/wavelength-division multiplexing and spatial multiplexing.

\subsection{Time-division multiplexing}
Time-division multiplexing makes use of specific time-windows in which it sends what data.
An example of this would be that the multiplexer sends data from one source for \qty{1}{\ms}, 
after which it sends the data from the next source for \qty{1}{\ms} and so on, untill it comes 
back to the data from the first source and it keeps repeating this cycle. Possible downsides of
this method are that both sides need synchronized clocks to acknowledgde when the data from 
each source is sent and the possibility of requiring a larger cache to store the data waiting 
for its timeslot to be sent

\subsection{Spatial multiplexing}
Spatial multiplexing keeps the data from each source physically separate from other data sent 
at the same time. Ways of achieving for optical communication include multifibre cables, 
multi-modal fibres and waveguides. Downsides of this technique mostly come down to high requirements 
for the transmitters, receivers and fibres themselves. Where time-division multiplexing is able to be 
sent through almost any optical fibre cable, spatial multiplexed signals set strict requirements, like 
multi-core cables which are thicker, special and experimental crystals at the transmitter, or optical cables 
without defects or minimal bends between transmitter and receiver.

\subsection{Wavelength-division multiplexing}
Wavelength-division multiplexing or WDM uses lasers of differing wavelengths between \qty{1270}{\nm} and 
\qty{1610}{\nm} to send multiple signals at once through an optical cable. If the frequencies are spaced 
correctly, with some room in between, they should not interfere with eachother. Usually this goes correctly, 
but since there are no physical or temporal divisions between the signals, this method does have one of the highest
chances of destructive interference among the previously mentioned methods. If the channel size is large 
enough, there is no problem. In practice the channel is about \qty{20}{\nm} in the least dense method, but new techniques 
allow more data to be send on smaller channels. A positive side of this method is that the requirements of the 
fibre optic are lower than for most spatial multiplexing methods.

