\section{background}
\label{sec:background}

The first step in designing a system is to explore what methods are currently 
available. A few of the main ways optical signals are currently multiplexed are: time-division 
multiplexing, frequency/wavelength-division multiplexing and spatial multiplexing.

\subsection{Time-division multiplexing}
Time-division multiplexing makes use of specific time-windows in which it sends what data.
An example of this would be that the multiplexer sends data from one source for \qty{1}{\ms}, 
after which it sends the data from the next source for \qty{1}{\ms} and so on, untill it comes 
back to the data from the first source and it keeps repeating this cycle. Possible downsides of
this method are that both sides need synchronized clocks to acknowledgde when the data from 
each source is sent and the possibility of requiring a larger cache to store the data waiting 
for its timeslot to be sent

\subsection{Spatial multiplexing}
Spatial multiplexing keeps the data from each source physically separate from other data sent 
at the same time. Ways of achieving for optical communication include multifibre cables, 
multi-modal fibres and waveguides. Downsides of this technique mostly come down to high requirements 
for the transmitters, receivers and fibres themselves. Where time-division multiplexing is able to be 
sent through almost any optical fibre cable, spatial multiplexed signals set strict requirements, like 
multi-core cables which are thicker, special and experimental crystals at the transmitter, or optical cables 
without defects or minimal bends between transmitter and receiver.

\subsection{Wavelength-division multiplexing}
Wavelength-division multiplexing or WDM uses lasers of differing wavelengths between \qty{1270}{\nm} and 
\qty{1610}{\nm} to send multiple signals at once through an optical cable. If the frequencies are spaced 
correctly, with some room in between, they should not interfere with eachother. Usually this goes correctly, 
but since there are no physical or temporal divisions between the signals, this method does have one of the 
highest chances of destructive interference among the previously mentioned methods. If the channel size is large 
enough, there is no problem. In practice the channel is about \qty{20}{\nm} in the least dense method (CWDM), 
but other techniques(DWDM) exist which allow more data to be send on smaller channels. A plus side of this method 
is that the requirements of the fibre optic cable are lower than for most spatial multiplexing methods.

\subsection{Optical Add/Drop Multiplexer}
A specific device which makes use of the functionalities of WDMs is the Optical Add/Drop Multiplexer(OADM). This 
device demultiplexes the signal, is able to remove one or more signals, and add some more signals to the stream. 
At the output of the OADM, the signal is multiplexed again and send on to the next node in the optic fibre cable.
Dependent on the specific construction of the Multiplexer, either O/E or straight optical, some devices are able 
to amplify the signal before sending it out again.

\subsection{Attenuation loss}
As mentioned in the OADM subsection, signals can be amplified by some devices. This is necessarry since the 
signal can lose its strength. The two main ways signals lose their strength is either loss by absorbtion 
or loss by scattering. The first loss is caused by the fact that the transport medium, usually a kind of glass, 
absorbs a part of the light. This kind of loss is usually expressed in dB/km, and preferably is as low as possible, 
especially for long distance communication. The second kind of loss is caused by bends in the cable. Light waves, 
prefer to travel in straight lines and when a bend occurs, some of the light will try to still go straight. This 
causes losses usually expressed in dB loss over a specific bend radius. 
