\section{Introduction}
\label{sec:introduction}
\IEEEPARstart {T}{his} document describes the specifications of the optical 
mulitplexer/demultiplexer Group 6 of advanced telecommunications will simulate 
reading a paper or PDF version of this document, please download the 
electronic file, trans\_jour.tex, from the IEEE Web site at \underline
{http://www.ieee.org/authortools/trans\_jour.tex} so you can use it to prepare your manuscript. If 
you would prefer to use LaTeX, download IEEE's LaTeX style and sample files 
from the same Web page. You can also explore using the Overleaf editor at 
\underline
{https://www.overleaf.com/blog/278-how-to-use-overleaf-with-}\discretionary{}{}{}\underline
{ieee-collabratec-your-quick-guide-to-getting-started\#.}\discretionary{}{}{}\underline{xsVp6tpPkrKM9}

If your paper is intended for a conference, please contact your conference 
editor concerning acceptable word processor formats for your particular 
conference. 

IEEE will do the final formatting of your paper. If your paper is intended 
for a conference, please observe the conference page limits. 

\subsection{Abbreviations and Acronyms}
Define abbreviations and acronyms the first time they are used in the text, 
even after they have already been defined in the abstract. Abbreviations 
such as IEEE, SI, ac, and dc do not have to be defined. Abbreviations that 
incorporate periods should not have spaces: write ``C.N.R.S.,'' not ``C. N. 
R. S.'' Do not use abbreviations in the title unless they are unavoidable 
(for example, ``IEEE'' in the title of this article).

\subsection{Other Recommendations}
Use one space after periods and colons. Hyphenate complex modifiers: 
``zero-field-cooled magnetization.'' Avoid dangling participles, such as, 
``Using \eqref{eq}, the potential was calculated.'' [It is not clear who or what 
used \eqref{eq}.] Write instead, ``The potential was calculated by using \eqref{eq},'' or 
``Using \eqref{eq}, we calculated the potential.''

Use a zero before decimal points: ``0.25,'' not ``.25.'' Use 
``cm$^{3}$,'' not ``cc.'' Indicate sample dimensions as ``0.1 cm 
$\times $ 0.2 cm,'' not ``0.1 $\times $ 0.2 cm$^{2}$.'' The 
abbreviation for ``seconds'' is ``s,'' not ``sec.'' Use 
``Wb/m$^{2}$'' or ``webers per square meter,'' not 
``webers/m$^{2}$.'' When expressing a range of values, write ``7 to 
9'' or ``7--9,'' not ``7$\sim $9.''

A parenthetical statement at the end of a sentence is punctuated outside of 
the closing parenthesis (like this). (A parenthetical sentence is punctuated 
within the parentheses.) In American English, periods and commas are within 
quotation marks, like ``this period.'' Other punctuation is ``outside''! 
Avoid contractions; for example, write ``do not'' instead of ``don't.'' The 
serial comma is preferred: ``A, B, and C'' instead of ``A, B and C.''

If you wish, you may write in the first person singular or plural and use 
the active voice (``I observed that $\ldots$'' or ``We observed that $\ldots$'' 
instead of ``It was observed that $\ldots$''). Remember to check spelling. If 
your native language is not English, please get a native English-speaking 
colleague to carefully proofread your paper.

Try not to use too many typefaces in the same article. You're writing
scholarly papers, not ransom notes. Also please remember that MathJax
can't handle really weird typefaces.

\subsection{Equations}
Number equations consecutively with equation numbers in parentheses flush 
with the right margin, as in \eqref{eq}. To make your equations more 
compact, you may use the solidus (~/~), the exp function, or appropriate 
exponents. Use parentheses to avoid ambiguities in denominators. Punctuate 
equations when they are part of a sentence, as in
\begin{equation}E=mc^2.\label{eq}\end{equation}

Be sure that the symbols in your equation have been defined before the 
equation appears or immediately following. Italicize symbols ($T$ might refer 
to temperature, but T is the unit tesla). Refer to ``\eqref{eq},'' not ``Eq. \eqref{eq}'' 
or ``equation \eqref{eq},'' except at the beginning of a sentence: ``Equation \eqref{eq} 
is $\ldots$ .''

\subsection{\LaTeX-Specific Advice}

Please use ``soft'' (e.g., \verb|\eqref{Eq}|) cross references instead
of ``hard'' references (e.g., \verb|(1)|). That will make it possible
to combine sections, add equations, or change the order of figures or
citations without having to go through the file line by line.

Please don't use the \verb|{eqnarray}| equation environment. Use
\verb|{align}| or \verb|{IEEEeqnarray}| instead. The \verb|{eqnarray}|
environment leaves unsightly spaces around relation symbols.

Please note that the \verb|{subequations}| environment in {\LaTeX}
will increment the main equation counter even when there are no
equation numbers displayed. If you forget that, you might write an
article in which the equation numbers skip from (17) to (20), causing
the copy editors to wonder if you've discovered a new method of
counting.

{\BibTeX} does not work by magic. It doesn't get the bibliographic
data from thin air but from .bib files. If you use {\BibTeX} to produce a
bibliography you must send the .bib files. 

{\LaTeX} can't read your mind. If you assign the same label to a
subsubsection and a table, you might find that Table I has been cross
referenced as Table IV-B3. 

{\LaTeX} does not have precognitive abilities. If you put a
\verb|\label| command before the command that updates the counter it's
supposed to be using, the label will pick up the last counter to be
cross referenced instead. In particular, a \verb|\label| command
should not go before the caption of a figure or a table.

Do not use \verb|\nonumber| inside the \verb|{array}| environment. It
will not stop equation numbers inside \verb|{array}| (there won't be
any anyway) and it might stop a wanted equation number in the
surrounding equation.

If you are submitting your paper to a colorized journal, you can use
the following two lines at the start of the article to ensure its
appearance resembles the final copy:

\smallskip\noindent
\begin{small}
\begin{tabular}{l}
\verb+\+\texttt{documentclass[journal,twoside,web]\{ieeecolor\}}\\
\verb+\+\texttt{usepackage\{\textit{Journal\_Name}\}}
\end{tabular}
\end{small}